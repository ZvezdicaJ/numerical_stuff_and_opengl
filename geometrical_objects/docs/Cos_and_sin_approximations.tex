\documentclass[openany, longbibliography,slovene,a4paper,12pt]{article}
%\documentclass[openany,slovene,a4paper,12pt]{article}
\usepackage[a4paper,inner=3.5cm,outer=2.5cm,top=2.5cm,bottom=2.5cm]{geometry}

\usepackage{braket}
\usepackage{float}
\usepackage{afterpage}
\usepackage{graphicx}
\usepackage{amssymb}

\usepackage[tbtags]{amsmath}
\usepackage[T1]{fontenc}
\graphicspath{{./slike/}{../slike_vezikel_z_robom/}{/home/jure/sola/magisterij/uporabljene_slike/}
{../eps_pdf/}}
\DeclareGraphicsExtensions{.eps,.jpeg,.png,.gif,.pdf}
\usepackage[outdir=./slike/]{epstopdf}
\epstopdfsetup{
	suffix=,
}
\usepackage[multidot]{grffile}

%\usepackage[slovene]{babel}      % slovenski delilni vzorci (!)
%\usepackage[english]{babel}
\usepackage[utf8]{inputenc}
\usepackage{makeidx}
\usepackage{enumerate}
\usepackage{caption}
\usepackage{subcaption}
\usepackage[tbtags]{mathtools}

\usepackage[section]{placeins}

\usepackage[hyphens,spaces,obeyspaces]{url}
\usepackage{breakurl}


\usepackage{ragged2e}
\edef\UrlBreaks{\do\-\UrlBreaks}

\usepackage{makeidx}
\pagestyle{headings}
\makeindex
\usepackage{fancyhdr}
\usepackage[titletoc,title]{appendix}


\usepackage[sort, numbers]{natbib}
\usepackage[pdfa]{hyperref}
\usepackage[x-1a]{pdfx}
\usepackage{pdfpages}
\usepackage{breqn}


\DeclareMathOperator{\arcsinh}{arcsinh}

\def\epsfg#1#2{\epsfig{file=#1.eps,width=#2}}
\def\legendamp#1#2{\vbox{\hsize=#1\caption{\small #2}}}

\setcounter{topnumber}{4}
\setcounter{bottomnumber}{4}
\setcounter{totalnumber}{5}
\renewcommand{\topfraction}{0.99}
\renewcommand{\bottomfraction}{0.99}
\renewcommand{\textfraction}{0.0}
\setlength{\tabcolsep}{10pt}
\renewcommand{\arraystretch}{1.5}

\def\bi#1{\hbox{\boldmath{$#1$}}}
\let\oldvec\vec
\def\vec#1{\mbox{\boldmath$#1$}}
\def\pol{{\textstyle{1\over2}}}
\def\svec#1{\mbox{{\scriptsize \boldmath$#1$}}}

\newcommand{\dif}{\mathrm{d}}
\usepackage{xparse}
\DeclareDocumentCommand{\myint}{o m o o}  
{%
	\int \IfValueT{#1}{#1} \dif #2 \IfValueT{#3}{\dif#3} \IfValueT{#4}{\dif#4}
}
\newcommand{\Alpha}{A}
\newcommand{\Beta}{B}
\newcommand{\Epsilon}{E}
\newcommand{\Kappa}{K}


\begin{document}
\section*{Chebyshev polynomials}
Chebyshev polynomials of the first kind are solutions to differential equation:
\begin{equation}
  (1-x^2)y''-xy'+n^2y=0.
  \end{equation}
  They are defined by recursion formula:
  \begin{equation}
    T_{n+1}(x)=2xT_n(x)-T_{n-1}(x) \quad \mathrm{with} \quad T_0(x)=0 \quad\mathrm{and}\quad T_1(x)=x.
  \end{equation}
On an interval $[-1,1]$ they form a complete basis set, similar to $\cos(x)$ and
$\sin(x)$ for  $x\in[-\pi, \pi]$. Of course one can expand these intervals to
arbitrary intervals $[-a,a]$ by a simple transformation $x\rightarrow x/a$.
Basis polynomials $T_n(x)$ are orthogonal with respect to the following scalar
product:
\[
  \braket{T_m(x)|T_n(x)}=\int_{-1}^{1}\frac{T_m(x)T_n(x)}{\sqrt{1-x^2}}\dif x = \begin{cases}
                                   \pi & n=m=0  \\
                                   \pi/2 & n=m \neq 0\\
                                   0 & n \neq m \\
  \end{cases}
\]
or, written for arbitrary interval $[-a, a]$:
\[
  \braket{T_m(x/a)|T_n(x/a)}=\int_{-a}^{a}\frac{T_m(x/a)T_n(x/a)}{\sqrt{1-(x/a)^2}}
  \frac{\dif x}{a} = \begin{cases}
                                   \pi & n=m=0  \\
                                   \pi/2 & n=m \neq 0\\
                                   0 & n \neq m \\
  \end{cases}
\].

One can now expand sin and cos functions into series on an interval $[-\pi,
\pi]$ using transformation $x\rightarrow x/\pi$:
\begin{equation}
  cos(x)=\sum_na_nT_n\left(\frac{x}{\pi}\right).
  \end{equation}
Multiplying both sides by $T_m(\frac{x}{\pi})/\sqrt{1-(x/\pi)^2}$ and
integrating over the interval $[-\pi, \pi]$ yields for $m\neq0$:
\begin{equation} \label{coeffm}
 \frac{1}{\pi}\int_{-\pi}^{\pi}\frac{\cos(x)T_m(\frac{x}{\pi})}{\sqrt{1-(x/\pi)^2}} \dif x = \frac{\pi}{2}a_m
\end{equation}
and for $m=0$:
\begin{equation}\label{coeff0}
 \frac{1}{\pi}\int_{-\pi}^{\pi}\frac{\cos(x)T_0(\frac{x}{\pi})}{\sqrt{1-(x/\pi)^2}} \dif x = \pi a_0.
\end{equation}

To simplify integral calculation we use relation between $T_n(x)$ and $\cos(x)$:
\begin{equation}
  T_n(\cos\theta)=\cos n\theta.
\end{equation}
Using $x/\pi = cos(z)$  in equations \label{coeffm} leads to:
\begin{equation}
  a_m = \frac{2}{\pi^2} \int_{-1}^{1}\frac{\cos(\pi\cos z)\cos(mz)}{\sqrt{1-\cos^2z}} \dif (\pi cosz)=-\frac{2}{\pi} \int_{\pi}^{0}\cos(\pi\cos z)\cos(mz) \dif z, 
  \end{equation}
where we took into account that $\cos z = -1$ when $z=\pi$ and $1$ when $z=-\pi$.
  For $m=0$ we get:
  \begin{equation}
    a_0 = \frac{1}{\pi^2} \int_{-1}^{1}\frac{\cos(\pi\cos z)}{\sqrt{1-\cos^2z}} \dif (\pi cos\theta)=-\frac{1}{\pi} \int_{\pi}^{0}\cos(\pi\cos z) \dif z.
  \end{equation}

  Chebyshev polynomial can be divided into even and odd polynomials. For even
  $n$ polynomial is even and for odd $n$ polynomial is odd. Thus the integral
  $\int cos(x) T_n(x\pi)$ will be zeo for all odd n. Only even $n$ polynomials
  will participate in expansion of $\cos(x)$ into a series using Chebyshev
  polynomials.
  The opposite happens when we expand $sin(x)$ into such series. Therefore only odd
  $n$ polynomials will participate in series expansion. For $m\neq0$ we get:

\begin{equation}
  a_m = \frac{2}{\pi} \int_{0}^{\pi}\sin(\pi\cos z)\cos(mz) \dif z, 
  \end{equation}
where we took into account that $\cos z = -1$ when $z=\pi$ and $1$ when $z=-\pi$.
  For $m=0$ we get:
  \begin{equation}
    a_0 = \frac{1}{\pi} \int_{0}^{\pi}\sin(\pi\cos z) \dif z.
  \end{equation}

\section{Legendre polynomials}
Legendre polynomials are solutions to a differential eqaution:
\begin{equation}
  \frac{\dif}{\dif x}\left[ (1-x^2)\frac{\dif P_n(x)}{\dif x} + n(n+1)P_n(x) \right] =0.
\end{equation}
They can also be obtained from a taylor series:
\begin{equation}
  \frac{1}{\sqrt{1-2xt+t^2}}=\sum_{n=0}^{\infty} P_n(x)t^n
  \end{equation}
Legendre polynomials are defined by equation:
\begin{equation}
  P_0 = 1 \quad P_1=1 \quad P_{n+1}(x)=\frac{2n+1}{n+1}xP_{n}(x) -  \frac{n}{n+1}P_{n-1}(x).
\end{equation}
They form an orthonormal basis set on the interval $[-1,1]$ with respect to the
following dot product:

\[   \braket{P_i, P_j}=\int_{-1}^{1} P_i(x) P_j(x) \dif x=
  \left\{
\begin{array}{ll}
      0 & i\neq j \\
      \frac{2}{2i+1} & \mathrm{otherwise} \\
\end{array} 
\right. \]


\section{Laguerre polynomials}
Laguerre poylnomials are a solution to the following differential equation:
\begin{equation}
  xy'' + (1-x)y'+ny=0,
  \end{equation}
where $n$ is non-negative integer. The generating function for these polynomials
is:
\begin{equation}
  \frac{1}{1-t}e^{-tx/(1-t)}=\sum_{n=0}^\infty t^nL_n(x).
  \end{equation}
  The polynomials can be calculated using derivation:
  \begin{equation}
    L_n(x)=\frac{e^x}{n!}\frac{\dif^n}{\dif x^n}\left(e^{-x}x^n \right)
    \end{equation}
    or recursive relation:
    \begin{equation}
      L_{k+1}(x)=\frac{(2k+1-x)L_k(x)-kL_{k-1}(x)}{k+1}.
    \end{equation}
    They are orthonormal with respect to the following scalar product:
    \[
      \braket{L_k(x), L_j(x)}= \int_0^\infty L_k(x)L_j(x)e^{-x} \dif x =
        \left\{
\begin{array}{ll}
      0 & k\neq j \\
      1 & k=j \\
\end{array} 
\right.
\]


\end{document}